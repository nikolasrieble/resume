%%%%%%%%%%%%%%%%%
% https://www.overleaf.com/latex/templates/altacv-template/trgqjpwnmtgv
% This is an sample CV template created using altacv.cls
% (v1.6.4, 13 Nov 2021) written by LianTze Lim (liantze@gmail.com). Now compiles with pdfLaTeX, XeLaTeX and LuaLaTeX.
%
%% It may be distributed and/or modified under the
%% conditions of the LaTeX Project Public License, either version 1.3
%% of this license or (at your option) any later version.
%% The latest version of this license is in
%%    http://www.latex-project.org/lppl.txt
%% and version 1.3 or later is part of all distributions of LaTeX
%% version 2003/12/01 or later.
%%%%%%%%%%%%%%%%

%% Use the "normalphoto" option if you want a normal photo instead of cropped to a circle
% \documentclass[10pt,a4paper,normalphoto]{altacv}

\documentclass[10pt,a4paper,ragged2e,withhyper]{altacv}
%% AltaCV uses the fontawesome5 and packages.
%% See http://texdoc.net/pkg/fontawesome5 for full list of symbols.

% Change the page layout if you need to
\geometry{left=1.25cm,right=1.25cm,top=1.5cm,bottom=1.5cm,columnsep=1.2cm}

% The paracol package lets you typeset columns of text in parallel
\usepackage{paracol}

% Change the font if you want to, depending on whether
% you're using pdflatex or xelatex/lualatex
\ifxetexorluatex
  % If using xelatex or lualatex:

  \usepackage{fontspec}
  \setmainfont{Futura}
  \renewcommand{\familydefault}{\sfdefault}
\else
  % If using pdflatex:
  \usepackage[rm]{roboto}
  \usepackage[defaultsans]{lato}
  %\usepackage{sourcesanspro}
  \renewcommand{\familydefault}{\sfdefault}
\fi

% Change the colours if you want to
\definecolor{Primary}{HTML}{11819C}

\colorlet{tagline}{Primary!80}
\colorlet{headingrule}{Primary!40}
\colorlet{accent}{Primary}


% Change some fonts, if necessary
% \renewcommand{\namefont}{\Huge\rmfamily\bfseries}
% \renewcommand{\personalinfofont}{\footnotesize}
% \renewcommand{\cvsectionfont}{\LARGE\rmfamily\bfseries}
% \renewcommand{\cvsubsectionfont}{\large\bfseries}


% Change the bullets for itemize and rating marker
% for \cvskill if you want to
\renewcommand{\itemmarker}{{\small\textbullet}}
\renewcommand{\ratingmarker}{\faCircle}

%% Use (and optionally edit if necessary) this .tex if you
%% want to use an author-year reference style like APA(6)
%% for your publication list
%% \input{pubs-authoryear}

%% Use (and optionally edit if necessary) this .tex if you
%% want an originally numerical reference style like IEEE
%% for your publication list
% \input{pubs-num}

%% sample.bib contains your publications
%% \addbibresource{sample.bib}

\begin{document}
\name{Nikolas Rieble}
\tagline{Software Engineer | Father}
%% You can add multiple photos on the left or right
\photoR{2.8cm}{../SmallPhoto.jpg}
% \photoL{2.5cm}{Yacht_High,Suitcase_High}

\personalinfo{%
  % Not all of these are required!
  \email{nikolas-rieble@protonmail.com}
  \location{Munich, Germany}
  \linkedin{nikolasrieble}
  \github{nikolasrieble}
  %% You can add your own arbitrary detail with
  %% \printinfo{symbol}{detail}[optional hyperlink prefix]
  % \printinfo{\faPaw}{Hey ho!}[https://example.com/]
  %% Or you can declare your own field with
  %% \NewInfoFiled{fieldname}{symbol}[optional hyperlink prefix] and use it:
  % \NewInfoField{gitlab}{\faGitlab}[https://gitlab.com/]
  % \gitlab{your_id}
  %%
  %% For services and platforms like Mastodon where there isn't a
  %% straightforward relation between the user ID/nickname and the hyperlink,
  %% you can use \printinfo directly e.g.
  % \printinfo{\faMastodon}{@username@instace}[https://instance.url/@username]
  %% But if you absolutely want to create new dedicated info fields for
  %% such platforms, then use \NewInfoField* with a star:
  % \NewInfoField*{mastodon}{\faMastodon}
  %% then you can use \mastodon, with TWO arguments where the 2nd argument is
  %% the full hyperlink.
  % \mastodon{@username@instance}{https://instance.url/@username}
}

\makecvheader
%% Depending on your tastes, you may want to make fonts of itemize environments slightly smaller
% \AtBeginEnvironment{itemize}{\small}

%% Set the left/right column width ratio to 6:4.
\columnratio{0.6}

% Start a 2-column paracol. Both the left and right columns will automatically
% break across pages if things get too long.
\begin{paracol}{2}
\cvsection{Experience}

\cvevent{Software Engineer}{Tulip}{July 2021 -- Ongoing}{Munich}
\begin{itemize}
  \item Enable customers to generate insights from data collected in the shop floor across production lines.
  \item Overhaul design of our machine monitoring product (Meteor, React, Redux)
  \item Adapt the development environment to run on the Apple silicone chip (Kubernetes, Bash, Docker, Postgres)
  \end{itemize}

\divider

\cvevent{Data Engineer}{Scalable Capital}{Oct. 2019 -- Jun. 2021}{Munich}
\begin{itemize}
\item Implemented ETL pipelines feeding a Data Warehouse from various sources (AWS, Salesforce).
\item Lead adoption of metabase a self-service BI tool with regular workshops and internal education.
\item Founding member of the clean code guild sharing best practice
\end{itemize}

\divider

\cvevent{Fullstack IoT Developer}{Hilti GmbH Industriegesellschaft für Befestigungstechnik}{Aug. 2018 - Sep. 2019}{Kaufering}
\begin{itemize}
\item Developed a Data Warehouse to offer a single source of truth across departments.
\item Introduced the team to version control with git and Atlassian Tools such as Jira, Confluence and Bamboo
\end{itemize}

\cvsection{Projects}

\cvevent{Android App: Einbürgerung Deutschland}{}{}{}
An \href{https://github.com/nikolasrieble/AndroidApp_GermanCitizenship}{open source} Android App with which the user can prepare themselves for the German Citizenship Test.
Published in the \href{https://play.google.com/store/apps/details?id=com.nrieble.quizapp}{Play Store}.

%\divider

%\cvevent{Operations Research}{}{}{}
%Processing Radioactive Waste in the US - A mathematical model (GAMS)

\medskip

\cvsection{A Day of My Life}

% Adapted from @Jake's answer from http://tex.stackexchange.com/a/82729/226
% \wheelchart{outer radius}{inner radius}{
% comma-separated list of value/text width/color/detail}
\wheelchart{1.5cm}{0.5cm}{%
  1/8em/accent!60/Read,
  2/8em/accent!60/Discuss solutions with colleagues,
  2/8em/accent!60/Mentor junior \\ engineers,
  4/8em/accent!60/Code,
  1/10em/accent/Freeletics,
  4/6em/accent!20/Spend time with family,
  1/6em/accent!20/Play board games \\ with friends
}

% use ONLY \newpage if you want to force a page break for
% ONLY the current column
\newpage

%% Switch to the right column. This will now automatically move to the second
%% page if the content is too long.
\switchcolumn

\cvsection{Life Philosophy}

\begin{quote}
''Empathy first''
\end{quote}

\cvsection{Most Proud of}

\cvachievement{\faHeartbeat}{My courage}{to leave my corporate job and join Scalable Capital right before my son was born}

\divider

\cvachievement{\faTrophy}{Graduation in psychology}{within three semesters of presence}

\divider

\cvachievement{\faBook}{Continous effort}{to become a software engineer}

\cvsection{Strengths}

\cvtag{Candid}
\cvtag{Attentive}
\cvtag{Solution-oriented}
\cvtag{Open}
\cvtag{Curious}

\divider\smallskip

\cvtag{Python}
\cvtag{Typescript}
\cvtag{React}
\cvtag{Kotlin}
\cvtag{SQL}
\cvtag{AWS}
\cvtag{TDD}
\cvtag{DDD}

\cvsection{Languages}

\cvskill{English}{5}
\cvskill{German}{5}
\cvskill{Turkish}{3.5}
\cvskill{Spanish}{2} %% Supports X.5 values.

%% Yeah I didn't spend too much time making all the
%% spacing consistent... sorry. Use \smallskip, \medskip,
%% \bigskip, \vspace etc to make adjustments.
\medskip

\cvsection{Education}

\cvevent{M.Sc.\ Computational Engineering Science}{Technical University Berlin}{2015 - 2018}{Berlin, Germany}

\divider

\cvevent{B.Sc.\ Psychology}{University of Vienna}{2013 - 2015}{Vienna, Austria}

\divider

\cvevent{B.Sc.\ Mechanical Engineering}{Technical University Stuttgart}{2009 - 2014}{Stuttgart, Germany}

% \divider

\end{paracol}


\end{document}
